
 \documentclass[letterpaper,11pt]{article}
 \usepackage[margin=1in,footskip=0.4in]{geometry}
\usepackage{amsmath,amssymb,amsfonts,mathrsfs,accents} %important math packages
\usepackage{enumerate} % package for making different lists
\usepackage[T1]{fontenc} % Encoding of fonts
\usepackage[utf8]{inputenc} % Encoding of input text
\usepackage[kerning]{microtype} % Better looking text
\usepackage[babel]{csquotes} % Better looking quotes
\usepackage{booktabs} % Better looking tables
\usepackage[american]{babel} % Language control, for hyphenation etc
\usepackage{lmodern}
\usepackage{graphicx} 
\usepackage{float} 
\usepackage{secdot}
\usepackage{mathtools}
\usepackage{placeins}
\usepackage{lscape}
\usepackage[hyphenbreaks]{breakurl}
\usepackage[hyphens]{url}
\usepackage{scrextend}
\usepackage{parskip}
\setlength\parindent{0pt}
\usepackage{setspace}
\doublespacing

%\usepackage{subcaption}
\usepackage{chngcntr}
\usepackage{wasysym}
\usepackage[explicit]{titlesec}
\usepackage{multirow}
\usepackage{subcaption}
\usepackage{caption}
\usepackage{bbm}
\usepackage{booktabs}
\usepackage{tabularx}
\usepackage{docmute} % to skip the preamble in tabout files
\usepackage{epstopdf}
\usepackage{upgreek} %write curly epsilon
\usepackage{bbold} %write indicator 1
%\usepackage[capposition=top]{floatrow}

\usepackage{longtable}
\usepackage{lscape}
\renewcommand{\baselinestretch}{1}

\usepackage[round]{natbib}

%%%%%%%%%%%%%%%%%%%%%%%%%%%%%%%%%%%%%%%%%%%%%%%%%%%%%%%%%%%%%%%%%%%%%%%%%%%%%%%
% Here's where the document begins
%
\begin{document}

\Large{\textbf{The Welfare Effects of Social Media}}

\small{\textit{With Hunt Allcott, Luca Braghieri and Matthew Gentzkow}}


%\end{flushright}
%\end{minipage}
 \bigskip

Abstract

\medskip

The rise of social media has provoked both optimism about potential societal benefits and concern about harms such as addiction, depression, and political polarization. In a randomized experiment, we find that deactivating Facebook for the four weeks before the 2018 US midterm election (i) reduced online activity, while increasing offline activities such as watching TV alone and socializing with family and friends; (ii) reduced both factual news knowledge and political polarization; (iii) increased subjective well-being; and (iv) caused a large persistent reduction in post-experiment Facebook use. Deactivation reduced post-experiment valuations of Facebook, suggesting that traditional metrics may overstate consumer surplus.


\end{document}
