
 \documentclass[letterpaper,11pt]{article}
 \usepackage[margin=1in,footskip=0.4in]{geometry}
\usepackage{amsmath,amssymb,amsfonts,mathrsfs,accents} %important math packages
\usepackage{enumerate} % package for making different lists
\usepackage[T1]{fontenc} % Encoding of fonts
\usepackage[utf8]{inputenc} % Encoding of input text
\usepackage[kerning]{microtype} % Better looking text
\usepackage[babel]{csquotes} % Better looking quotes
\usepackage{booktabs} % Better looking tables
\usepackage[american]{babel} % Language control, for hyphenation etc
\usepackage{lmodern}
\usepackage{graphicx} 
\usepackage{float} 
\usepackage{secdot}
\usepackage{mathtools}
\usepackage{placeins}
\usepackage{lscape}
\usepackage[hyphenbreaks]{breakurl}
\usepackage[hyphens]{url}
\usepackage{scrextend}
\usepackage{parskip}
\setlength\parindent{0pt}
\usepackage{setspace}
\doublespacing

%\usepackage{subcaption}
\usepackage{chngcntr}
\usepackage{wasysym}
\usepackage[explicit]{titlesec}
\usepackage{multirow}
\usepackage{subcaption}
\usepackage{caption}
\usepackage{bbm}
\usepackage{booktabs}
\usepackage{tabularx}
\usepackage{docmute} % to skip the preamble in tabout files
\usepackage{epstopdf}
\usepackage{upgreek} %write curly epsilon
\usepackage{bbold} %write indicator 1
%\usepackage[capposition=top]{floatrow}

\usepackage{longtable}
\usepackage{lscape}
\renewcommand{\baselinestretch}{1}

\usepackage[round]{natbib}

%%%%%%%%%%%%%%%%%%%%%%%%%%%%%%%%%%%%%%%%%%%%%%%%%%%%%%%%%%%%%%%%%%%%%%%%%%%%%%%
% Here's where the document begins
%
\begin{document}

\Large{\textbf{Broadband Internet Subsidies and Educational Success}}



%\end{flushright}
%\end{minipage}
 \bigskip

Abstract

\medskip

What is the impact of internet access at home on low-income students' success in school? I evaluate a large broadband subsidy program for low-income households in the US, Comcast Internet Essentials, to answer this question. The exact timing of the introduction of the program, as well as sharp eligibility thresholds, variation in Comcast availability, and numerous expansions to eligibility criteria over time provide rich sources of quasi-exogenous variation in households' propensities to adopt broadband internet. I leverage this variation to estimate the causal effect of broadband internet at home on student performance in standardized tests. Reduced form evidence using data on the average test scores in math and reading for the universe of public school districts in the US suggests that broadband internet significantly improves reading skills, while having little impact on math skills. Effects are particularly pronounced for children from households where English is not the primary language.


\end{document}
